\section{}

\subsection{second approach}

we use the definition of the Lambert W function 
\(W(x)e^{W(x)}=x\) to derive the following identity

\begin{align}
	     \frac{1}{W(\ln(x))}=\frac{e^{W(\ln(x))}}{\ln(x)}
\end{align}

Now we use the generalised Leibniz rule

\begin{align}
	     \frac{d^{\xi}}{dx^{\xi}}\frac{e^{W(\ln(x))}}{\ln(x)}=
	     \sum_{a+b=\xi}\binom{\xi}{a, b}\frac{d^a}{dx^a}e^{W(\ln(x))}
	     \frac{d^b}{dx^b}\frac{1}{\ln(x)}
\end{align}

Then we divide this problem into smaller sub-problems as

\begin{align}
	     \mathcal{G}_1 &= \frac{d^b}{dx^b}\frac{1}{\ln(x)} \\
	     \mathcal{G}_2 &= \frac{d^a}{dx^a}e^{W(\ln(x))}
\end{align}

\subsubsection{G - 1}

Let \(\gamma(\xi, x) :=\frac{d^\xi}{dx^\xi}\frac{1}{\ln(x)}\)

\begin{align}
	     \gamma(1, x) &= \frac{-1}{x\ln^2(x)} \\
	     \gamma(2, x) &= \frac{\ln(x)+2}{x^2\ln^3(x)} \\
	     \gamma(3, x) &= \frac{-2(\ln^2(x)+3\ln(x)+3)}{x^3\ln^4(x)} \\
	     \gamma(4, x) &= \frac{6\ln^3(x)+22\ln^2(x)+36\ln^2(x)+24}{x^4\ln^5(x)} 
\end{align}

We can establish the hypothesis:

\begin{align}
    \gamma(\xi, x) &= \frac{(-1)^\xi}{x^\xi\ln^{\xi+1}}
    \sum_{t_2=0}^\xi \ln^{t_2}(x)h_{\xi, t_2}
\end{align}

\begin{table}[h!]
	     \centering
	     \begin{tabular}{||c| c c c c c||} 
	     	     \hline
	     	     \hline
	     	     h_{\xi, t_2}& \(t_2=1\) & \(t_2=2\) & \(t_2=3\)& \(t_2=4\)& \(t_2=5\) \\ [0.5ex] 
	     	     \hline
	     	     \xi=1 & 1 &&&& \\ 
	     	     \xi=2 & 1 & 2 &&& \\
	     	     \xi=3 & 2 & 6 & 6 &&\\
	     	     \xi=4 & 6 & 22 & 36 & 24 &\\
	     	     \xi=5 & 24 & 100 & 210 & 240 & 120\\[1ex] 
	     	     \hline
	     	     \hline
 	     \end{tabular}
\end{table}

\begin{align}
	     h_{\xi, t_2} = (\xi-t_2+1)!|S_1(t_2, t_2-\xi+1)|
\end{align}

Thus

\begin{align}
	     \gamma(\xi, x) &= \frac{(-1)^\xi}{x^\xi\ln^{\xi+1}}
	     \sum_{t_2=0}^\xi \ln^{t_2}(x)(\xi-t_2+1)!|S_1(t_2, t_2-\xi+1)|
\label{nnnn}
\end{align}

\subsubsection{G - 2}

we take a higher derivative and extend the relationship by the sum

\begin{align}
	     \frac{d^\xi}{dx^\xi}e^{W(\ln(x))} =e^{W(\ln(x))}
	     \sum_{k=0}^\xi \frac{1}{k!}\sum_{j=0}^k (-1)^j \binom{k}{j}W^j(\ln(x))
	     \frac{d^\xi}{dx^\xi}W^{k-j}(\ln(x))
\end{align}

in order to find an explicit solution we must find an 
explicit relationship for the following formula.
\(\frac{d^\xi}{dx^\xi}W^{k-j}(\ln(x))\). First we will
try to find this pattern through observation. Let 
\(\Phi(n, x, \xi) = \frac{d^\xi}{dx^\xi}W^{n}(\ln(x))\equiv \Phi(\xi)\). Then

\begin{align}
	     \Phi(1) &= \frac{nW(\ln(x))^n}{x\ln(x)W(\ln(x))+x\ln(x)}\\
	     \Phi(2) &= \frac{nW(\ln(x))^n(-\ln(x)+1)W(\ln(x))^2+(n-2\ln(x)-3)W(\ln(x))-(\ln(x)+1-n)}{(x^2\ln^2(x)(W(\ln(x))+1)^3}
\end{align}

if we generalize it

\begin{align}
	     \Phi(n, x, \xi) = \frac{nW(\ln(x))^n}{x^\xi\ln^\xi(x)(W(\ln(x))+1)^{2\xi-1}}M(\xi, n)
\end{align}

where

\begin{align}
	     M(\xi, n) = \sum_{l=0}^{2\xi-1}W^l(\ln(x))(-1)^{l+1}N(\xi, n, l)
\end{align}

and

\begin{align}
	     N(\xi, n, l) = \sum_{t_1=0}^\xi \ln^{t_1}(x)c^1_{t_1, \xi, n , l} +
	     \sum_{t_2=1}^{\xi-1}nc^2_{t_2, \xi, n, l}+\sum_{t_3=1}^{\xi-2}\ln(a)nc^3_{t_3, \xi, n, l}
\end{align}

\newpage

for \(c^1_{t_1, \xi, n, l}\):

\begin{table}[h!]
	     \centering
	     \begin{tabular}{||c| c ||} 
 	     	     \hline
	     	     \hline
	     	     \(\xi = 1\) & \(t_1=1\) \\ [0.5ex] 
	     	     \hline
	     	     l=1 & 1 \\ [1ex] 
	     	     \hline
	     	     \hline
	     \end{tabular}
\end{table}

\begin{table}[h!]
	     \centering
	     \begin{tabular}{||c| c c ||} 
	     	     \hline
	     	     \hline
	     	     \(\xi = 2\) & \(t_1=1\) & \(t_1=2\) \\ [0.5ex] 
	     	     \hline
	     	     l=1 & 1 & 1 \\ 
	     	     l=2 & 3 & 2\\
	     	     l=3 & 1 & 1\\[1ex] 
	     	     \hline
	     	     \hline
	     \end{tabular}
\end{table}

\begin{table}[h!]
	     \centering
	     \begin{tabular}{||c| c c c||} 
	     	     \hline
	     	     \hline
	     	     \(\xi = 2\) & \(t_1=1\) & \(t_1=2\) & \(t_1=3\) \\ [0.5ex] 
	     	     \hline
	     	     l=1 & 2 & 3 & 2\\ 
	     	     l=2 & 11 & 15 & 8\\
	     	     l=3 & 20 & 24 & 12\\
	     	     l=4 & 10 & 15 & 18\\[1ex] 
	     	     \hline
	     	     \hline
	     \end{tabular}
\end{table}

for \(c^2_{t_2, \xi, n, l}\):

\begin{table}[h!]
	     \centering
	     \begin{tabular}{||c| c ||} 
	     	     \hline
	     	     \hline
	     	     \(\xi = 1\) & \(t_2=1\) \\ [0.5ex] 
	     	     \hline
	     	     l=1 & 1 \\[1ex] 
	     	     \hline
	     	     \hline
	     \end{tabular}
\end{table}

\begin{table}[h!]
	     \centering
	     \begin{tabular}{||c| c ||} 
	     	     \hline
	     	     \hline
	     	     \(\xi = 1\) & \(t_2=1\) \\ [0.5ex] 
	     	     \hline
	     	     l=1 & 1 \\
	     	     l=2 & 1 \\[1ex] 
	     	     \hline
	     	     \hline
	     \end{tabular}
\end{table}

\begin{table}[h!]
	     \centering
	     \begin{tabular}{||c| c c ||} 
	     	     \hline
	     	     \hline
	     	     \(\xi = 2\) & \(t_2=1\) & \(t_2=2\) \\ [0.5ex] 
	     	     \hline
	     	     l=1 & 3 & 0 \\ 
	     	     l=2 & -12 & 1\\
	     	     l=3 & -12 & 2\\[1ex] 
	     	     \hline
	     	     \hline
	     \end{tabular}
\end{table}

for \(c^3_{t_3, \xi, n, l}\):

\begin{table}[h!]
	     \centering
	     \begin{tabular}{||c| c  ||} 
	     	     \hline
	     	     \hline
	     	     \(\xi = 2\) & \(t_2=1\) \\ [0.5ex] 
	     	     \hline
	     	     l=1 & -3  \\ 
	     	     l=2 & -9 \\
	     	     l=3 & -9 \\[1ex] 
	     	     \hline
	     	     \hline
	     \end{tabular}
\end{table}

This solution procedure also leads nowhere.
