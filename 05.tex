\section{Analytický způsob}

Využiji kompletně analytický neoriginální postup,
aby jsem dostal alespoň něco. Takže máme
\(ssrt(x)=e^{W(\ln(x)}\), potom využiji mocninnou 
expanzi exponenciální funkce

\begin{align}
        e^x = \sum_{k=0}^\infty \frac{x^k}{k!}
\end{align}

takže dostaneme

\begin{align}
        e^{W_0(\ln(x)} = \sum_{k=0}^{
        \infty}\frac{W_0^k(\ln(x))}{k!}
\end{align}

potom aplikuji sérii, která reprezentuje
principiální větev Lambertovy W funkce:

\begin{align}
        W_0(x) = \sum_{k=0}^\infty 
        \frac{(-n)^{n-1}}{n!}x^n
\end{align}

toto umocním na k-tou mocninu, jo ehm 
ta 0 značí, že děláme v principiální
větvi Lambertovy W funkce

\begin{align}
        W_0^k(\ln(x)) = \left(\sum_{k=
        0}^\infty \frac{(-n)^{n-1}}{n!}(
        \ln(x))^n\right)^k
\end{align}

potom využiji multinomiální formuli

\begin{align}
        W_0^k(\ln(x)) &= \left(\sum_{k=0}^\infty 
        \frac{(-n)^{n-1}}{n!}\ln(x)^n\right)^k \\
        &= \sum_{n=0}^\infty \left(\sum_{\substack{0
        \leq r_1 , ..., r_k <n \\ r_1 + ... + r_k = 
        n}}\frac{k!}{r_1 ... r_k}\prod_{t=1}^n \frac{(
        -r_t)^{r_t - 1}}{r_t !} \right)\ln^n(x)
\end{align}

takže

\begin{align}
        ssrt(x) = e^{W_0(\ln(x)} = 
        \sum_{k=0}^{\infty}\frac{1}{k!}
        \sum_{n=0}^\infty \left(\sum_{
        \substack{0\leq r_1 , ..., r_k 
        <n \\ r_1 + ... + r_k = n}}\frac
        {k!}{r_1 ... r_k}\prod_{t=1}^n 
        \frac{(-r_t)^{r_t - 1}}{r_t !} 
        \right)\ln^n(x)
\end{align}

pro \((e^{-1/e}, e^{1/e})\), jelikož jsem zapomněl 
zmínit, že Sériová reprezentace Lambertovy W funkce 
je konvergentní pravě tehdy, když definiční obor W 
funkce je v tomto intervalu \(|x|<1/e\). My ale 
nepracujeme s W(x) ale \(W(\ln(x))\), takže,
interval upravím na \(|\ln(x)|<1/e\), což lze 
rozepsat na \(-1/e < \ln(x) < 1/e\), zbavím se
přirozeného logaritmu \((e^{-1/e}, e^{1/e})\) a
máme interval, o kterém jsem mluvil. Následně 
aplikujeme tento vztah pro přirozený logaritmus

\begin{align}
        \ln(x) = \sum_{k=1}^\infty 
        \frac{(-1)^{k-1}(x-1)^k}{k}
\end{align}

což platí pro \(|x-1|<1\) a \(x \neq 0\), V 
našem vztahu, ale využívám přirozený logaritmus
na n-tou, potom opět aplikuji multinomiální 
pravidlo.

\begin{align}
        \ln^n(x) = \left(\sum_{k=1}^\infty 
        \frac{(-1)^{k-1}(x-1)^k}{k}\right)^n
\end{align}
\begin{align}
        \ln^n(x)
        &= \sum_{w=0}^\infty \left(\sum_{
        \substack{0\leq r_1 , ..., r_n <w 
        \\ r_1 + ... + r_n = w}}\frac{n!}{
        r_1 ... r_k}\prod_{t=1}^w \frac{(-1
        )^{r_t-1}}{r_t} \right)(x-1)^w
\end{align}

toto dáme do finálního vztahu

\begin{align}
        ssrt(x) = \sum_{k=0}^{\infty}\frac{1}{k!}\left[
        \sum_{n=0}^\infty \left[\left(\sum_{
        \substack{0\leq r_1 , ..., r_k <n \\ 
        r_1 + ... + r_k = n}}\frac{k!}{r_1 ...
        r_k}\prod_{t=1}^n \frac{(-r_t)^{r_t 
        - 1}}{r_t !} \right) \sum_{w=0}^\infty
        \left[ \left(\sum_{\substack{0\leq r_1
        , ..., r_n <w \\ r_1 + ... + r_n = w}}
        \frac{n!}{r_1 ... r_n}\prod_{t=1}^w
        \frac{(-1)^{r_t-1}}{r_t} \right)(x-1)^w
        \right]\right]\right]
\end{align}

což přepíšeme na

\begin{align}
        ssqrt(x) = \sum_{w=0}^\infty (x-1)^w
        \sum_{k=w}^\infty \frac{1}{k!}\sum_{n=w}^\infty
        \left(\sum_{\substack{0\leq r_1 , ..., r_k <n
        \\ r_1 + ... + r_k = n}}\frac{k!}{r_1 ... r_k}
        \prod_{t=1}^n \frac{(-r_t)^{r_t - 1}}{r_t !} 
        \right)\left(\sum_{\substack{0\leq r_1 , ..., 
        r_n <w \\ r_1 + ... + r_n = w}}\frac{n!}{r_1 
        ... r_n}\prod_{t=1}^w \frac{(-1)^{r_t-1}}{r_t}
        \right)  
\end{align}

a to je mocninná expanze super-odmocniny pro 
\(x\in (e^{-1/e}0) \cup (0,  e^{1/e})\). Když 
toto porovnáme s 

\begin{align}
        ssrt(x) =1+(x-1)-(x-1)^{2}+\frac{3}{2}(x-1)^{3}
        \frac{17}{6}(x-1)^{4}+\frac{37}{6}(x-1)^{5}-\frac
        {1759}{120}(x-1)^{6}+\frac{13279}{360}(x-1)^{7}+O((x-1)^8)
\end{align}

z čehož vytvoříme tuto sumu

\begin{align}
        ssrt(x) = \sum_{w=0}^\infty c_w (x-1)^w
\end{align}

potom koeficienty jsou dány vztahem

\begin{align}
        c_w = \sum_{k=w}^\infty \frac{1}{k!}\sum_{n=w}^\infty 
        \left(\sum_{\substack{0\leq r_1 , ..., r_k <n \\ r_1 + 
        ... + r_k = n}}\frac{k!}{r_1 ... r_k}\prod_{t=1}^n
        \frac{(-r_t)^{r_t - 1}}{r_t !} \right)\left(\sum_{
        \substack{0\leq r_1 , ..., r_n <w \\ r_1 + ... + r_n 
        = w}}\frac{n!}{r_1 ... r_n}\prod_{t=1}^w \frac{(-1)^{r_t-1}}{r_t} \right)  
\end{align}

Dokázali jsme to!
